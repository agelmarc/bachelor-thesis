To assess our neural network extrapolation method, we want to compare it to classical, state-of-the-art model-space extrapolations. Classical extrapolations are usually done using exponential fits. Even though the convergence of ground state energy sequences does not resemble an exponential convergence, it provides a good estimate on the ground-state energy. To ensure comparability, we resrict the classical extrapolation method to the same $N_\mathrm{max}$ values as in the neural network extrapolation. That means that input sequences are formatted exactly as described earlier, using all subsets of 3 out of all available oscillator frequencies.

In a classical extrapolation, an exponential function
\begin{equation}
  f(x) = a \exp(-bx) + c
\end{equation}
is fitted on a sequence of three $N_\mathrm{max}$ values with a specific oscillator frequency. Out of all available frequencies, we choose the frequency which has the lowest energy value at the highest $N_\mathrm{max}$. This is motivated by the fact that these sequences are usually more converged, which ensures more precision in the function fit. For example, if we wanted to extrapolate the sample in \autoref{fig:example_evaluation} with frequencies $\hbar\Omega = 16, 20, 24$ for the case of maximum $N_\mathrm{max}$ of 12, we choose the values  $N_\mathrm{max} = 8, 10, 12$ of the $\hbar\Omega = 16$ sequence for extrapolation, since the absolute value of its sequence is the lowest among the three frequencies at $N_\mathrm{max} = 12$.
