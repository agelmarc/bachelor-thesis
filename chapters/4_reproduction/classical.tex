In this section, we want to compare our extrapolation framework to a classical approach of extrapolating NCSM sequences. For that, exponential function fits as described in \autoref{sec:classical_extrapolation} are used to extrapolate the limit value of the sequences. The resulting extrapolation values are shown as red tickmarks in \autoref{fig:eval_vanilla} for each maximum $N_\mathrm{max}$.

We first look at the exponential extrapolations of \n{3}{H} and \n{4}{He}. For all $N_\mathrm{max}$, the exponential fit results in a much smaller uncertainty of the fitted limits. Furthermore, the exponential extrapolations do not violate the variational boundary condition for those $N_\mathrm{max}$. Generally, the exponential extrapolations lie way closer to the variational boundary than the extrapolations from our framework. Since the evaluated sequences converge fast (the convergence can already be seen for the higher oscillator frequencies at the higher $N_\mathrm{max}$ values), the variational bound gives a good estimate on the actual ground-state energy and thus, the exponential seem to be very accurate. This is reinforced by the fact that the exponential fits get more precise for the higher SRG flow parameter. This can especially be seen in the $N_\mathrm{max}=8$ point for \n{4}{He}. For the nuclei \n{3}{H} and \n{4}{He}, the exponential fits clearly outperform our extrapolation framework for all $N_\mathrm{max}$.

Considering the fact that the NCSM calculations, from which those extrapolations base off, already converge fast, those nuclei provide ideal conditions for the exponential extrapolations. As such, those nuclei do not represent an actual use case for extrapolations, since they are not needed for converged NCSM results. To better compare the two extrapolation methods, we look at the extrapolations of \n{2}{H}, which provides a more realistic use case, as the NCSM results do not converge as fast. Here, we see that both extrapolation methods are very comparable, both in precision and in accuracy. Interestingly, even the classical extrapolations violate the variational boundary condition for all $N_\mathrm{max}$ except $N_\mathrm{max} = 10$ for \srg{0.04}. This seems unlogical at first, since exponential fits should naturally enforce a monotonicity constraint on the extrapolated limit. Actually, the exponential fits that result in a too high prediction are coming from the sequences, for which the sequences have not converged as much. Since the variational boundary is determined by the frequencies for which the sequence has converged the most, it is possible for some sequences to result in a prediction which violates the variational boundary condition. This indicates that the violation of the network extrapolations do not come from an unoptimized training with unbalanced input sequences, but from the fact that the sequences are slow to converge. Also, NCSM calculations for the ground-state energy generally do not show an exponential convergence, which will lead to an increased uncertainty for lower $N_\mathrm{max}$. Since the exponential fits force a lower limit than the value at the highest $N_\mathrm{max}$, it is guaranteed that at least one prediction is lower than the variational boundary.

An important difference between the exponential fits and our extrapolation framework is a preconditioning of some sort for the input sequences. For the exponential function fits, this preconditioning consists of only using the best sequence, i.e. the sequence that has the lowest energy value at the highest $N_\mathrm{max}$, for the extrapolation. In our extrapolation framework, we intentionally do not preselect the "best" sequences in order to provide a more general extrapolation tool that does not depend on any sort of preselecting, which requires a metric on what sequences to pick for extrapolating the ground-state energy. In that regard, our extrapolation framework produces promisable results for unconverged sequences, as they are already comparable in quality to the classical extrapolations.
