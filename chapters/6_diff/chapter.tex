In our basic extrapolation framework, we trained networks on predicting the exact ground-state energy given a few sequences of absoulte energy values. This method is inherently dependent on the absolute values of the energy, since the networks fundamentally consist of linear maps from one layer to the next. Nuclear ground-state energies have a very broad range, which we already saw in our inspected nuclei. For example, \n{2}{H} has a ground-state energy of approximately \SI{-2.1}{\mega\electronvolt} and \n{4}{He} has a ground-state energy of approximately \SI{-28.4}{\mega\electronvolt}, based off the NCSM calculations shown in \autoref{fig:example_nmax} and \autoref{fig:eval_extended}. As a result, we cannot confidentially exclude the possibility of the absolute ground-state energies influencing the network predictions.

In this chapter, we want to analyze the influence of the absolute energies on the network prediction by implementing a different extrapolation method based on our previous framework, which will work by extrapolating the differences of the energy sequences.
\section{framework}
In order to implement our difference based extrapolation framework, we have to think about how we have to change the generation of the input data, the network structure as well as the extrapolation process of our basic extrapolation framework.

We want to base our network training on the same data set as the basic framework. The networks now should get the differences of consecutive ground-state energies in the sequences as an input. To achieve this, we generate the same formatted input sequences out of the raw NCSM calculations, using the inflation algorithm described in \autoref{sec:inflate}. The formatted input data thus consists of four consecutive energy values for three oscillator frequencies, as well as a limit which will be used to compute the prediction target of our networks. To turn those sequences into our input difference sequences, we calculate the three differences between the four energy values. When the networks get input data in the form of energy differences, they cannot possibly predict an absolute ground-state energy from them. This means that the prediction target will also have to be modified to a difference. It is natural to take the difference between the last value of the energy sequence and the limit as a target. However, this leads to a problem, as the limit is the same for the three oscillator frequencies for a given nucleus and interaction, but the last values of the energy sequences can vary depending on the frequency. Because the three difference sequences are all put into a network simultaneously, we have to find a single common extrapolation target. For this, we decide to take the difference between the mean of the last values for all three sequences and the limit.

Note that, since we now only have three direct inputs available, which are the differences of the four absolute values, the network structure has to be adjusted to be compatible with this data. For the input layer, we have $L^{(1)} = 3 \times 3 = 9$ input neurons. Furthermore, we reduce some of the hidden neurons in the hidden layers to accommodate the reduced input neurons. We empirically decide on $L^{(2)} = 18$, $L^{(3)} = 12$. The single output neuron stays the same. We could have chosen to keep the network structure the same, but this would require to input four differences of five energies, which would mean that those networks got more information about the sequences. To ensure comparability, we chose to keep the information the same but adjust the network structure.

Further modifications include the adjustment of our sequence shifting. In our basic framework, we decided to shift the formatted sequences by a random amount of $[\SI{-10}{\mega\electronvolt}, \SI{10}{\mega\electronvolt}]$ to counter the dependency on absolute values. They are no longer needed, as they would cancel out when calculating the differences. Also, the threshold for deeming a network valid is reduced to \SI{0.1}{\mega\electronvolt}, as the energy differences which are input are generally much smaller than the absolute values.

To compensate for the shift in the training set, we also have to modify the evaluation of a network. The evaluation will work the same way as in our basic framework, but now, the final prediction is calculated by adding the mean of the last energy values back onto the network prediction.

\section{results and comparison}
Using our newly constructed difference based extrapolation framework, we can now do the same analysis on our three nuclei \n{2}{H}, \n{3}{H} and \n{4}{He} using the same interaction as used earlier. Even though the difference based framework uses differences to compute a difference prediction internally, the final evaluation will still base off the same absolute energy differences and result in an absolute energy prediction. This means that we can apply our different evaluation modifications from \autoref{chap:extended} and discuss the influences of them on the difference based evaluation. Furthermore, using the same nuclei allows us to compare the difference based extrapolation and the absolute value extrapolation.

The results of our difference based extrapolation is shown in \autoref{fig:eval_diff}. The final extrapolated values and their uncertainty are shown in \autoref{tab:eval_diff} as well for each $N_\mathrm{max}$ and each SRG flow parameter.

% Generell: Predictions VIEL genauer, besonders bei hohem Nmax
% Generell: Trainingsmodes machen nicht mehr viel aus
% Kein unterbinden mehr -> "monoton steigende predictions" sind nicht mehr da
% He4 0.04 vs 0.08: Effekt von sehr unterschiedlichen sequenzen viel größer als bei abs (abs besser)
% 0.08: SRG hat viel mehr auswirkung als bei abs und bei 0.04
% H2: 0.08 schlechter als 0.04

\begin{table}[H]
  \caption{Extrapolation results in \si[]{\mega\electronvolt} of the difference based framework for the nuclei \n{2}{H} \textbf{(a)}, \n{3}{H} \textbf{(b)} and \n{4}{He} \textbf{(c)}. For each interaction characterized by the flow parameter $\alpha = \srg{0.04}, \srg{0.08}$, the final extrapolation results for the given $N_\mathrm{max}$ value is shown. Here, \textbf{(1)} is our basic extrapolation without further modifications of the training process, \textbf{(2)} is the $N_\mathrm{max}$-limitation training mode, \textbf{(3)} is the SRG-filter training mode. }
  \label{tab:eval_diff}
  \centering
  \begin{subtable}{\textwidth}
    \caption{}
    \centering
    \begin{tabular}{
        r|
        S[table-format=-2.3(3)]
        S[table-format=-2.3(3)]
        S[table-format=-2.3(3)]
        S[table-format=-2.3(3)]
        S[table-format=-2.3(3)]
        S[table-format=-2.3(3)]
      }
      \toprule
      $\alpha$                         &
      \multicolumn{3}{c}{$\srg{0.04}$} &
      \multicolumn{3}{c}{$\srg{0.08}$}   \\
      \midrule
      $N_\mathrm{max}$                 &
      {8}                              &
      {10}                             &
      {12}                             &
      {8}                              &
      {10}                             &
      {12}                               \\
      \midrule
      (1)                              &
      -2.066 \pm 0.065                 &
      -2.144 \pm 0.025                 &
      -2.129 \pm 0.022                 &
      -2.054 \pm 0.069                 &
      -2.113 \pm 0.028                 &
      -2.126 \pm 0.023                   \\
      (2)                              &
      -2.080 \pm 0.055                 &
      -2.146 \pm 0.023                 &
      -2.127 \pm 0.020                 &
      -2.064 \pm 0.056                 &
      -2.110 \pm 0.026                 &
      -2.124 \pm 0.022                   \\
      (3)                              &
      -2.068 \pm 0.055                 &
      -2.153 \pm 0.038                 &
      -2.149 \pm 0.034                 &
      -2.062 \pm 0.041                 &
      -2.124 \pm 0.032                 &
      -2.158 \pm 0.036                   \\
      \bottomrule
    \end{tabular}
  \end{subtable}
  \par\bigskip
  \begin{subtable}{\textwidth}
    \caption{}
    \centering
    \begin{tabular}{
        r|
        S[table-format=-2.3(3)]
        S[table-format=-2.3(3)]
        S[table-format=-2.3(3)]
        S[table-format=-2.3(3)]
        S[table-format=-2.3(3)]
        S[table-format=-2.3(3)]
      }
      \toprule
      $\alpha$                         &
      \multicolumn{3}{c}{$\srg{0.04}$} &
      \multicolumn{3}{c}{$\srg{0.08}$}   \\
      \midrule
      $N_\mathrm{max}$                 &
      {8}                              &
      {10}                             &
      {12}                             &
      {8}                              &
      {10}                             &
      {12}                               \\
      \midrule
      (1)                              &
      -8.453 \pm 0.099                 &
      -8.528 \pm 0.044                 &
      -8.470 \pm 0.026                 &
      -8.410 \pm 0.090                 &
      -8.465 \pm 0.035                 &
      -8.460 \pm 0.025                   \\
      (2)                              &
      -8.461 \pm 0.082                 &
      -8.526 \pm 0.036                 &
      -8.468 \pm 0.025                 &
      -8.422 \pm 0.074                 &
      -8.464 \pm 0.031                 &
      -8.457 \pm 0.023                   \\
      (3)                              &
      -8.390 \pm 0.064                 &
      -8.493 \pm 0.028                 &
      -8.466 \pm 0.026                 &
      -8.408 \pm 0.045                 &
      -8.460 \pm 0.026                 &
      -8.466 \pm 0.025                   \\
      \bottomrule
    \end{tabular}
  \end{subtable}
  \par\bigskip
  \begin{subtable}{\textwidth}
    \caption{}
    \centering
    \begin{tabular}{
        r|
        S[table-format=-2.3(3)]
        S[table-format=-2.3(3)]
        S[table-format=-2.3(3)]
        S[table-format=-2.3(3)]
        S[table-format=-2.3(3)]
        S[table-format=-2.3(3)]
      }
      \toprule
      $\alpha$                         &
      \multicolumn{3}{c}{$\srg{0.04}$} &
      \multicolumn{3}{c}{$\srg{0.08}$}   \\
      \midrule
      $N_\mathrm{max}$                 &
      {8}                              &
      {10}                             &
      {12}                             &
      {8}                              &
      {10}                             &
      {12}                               \\
      \midrule
      (1)                              &
      -28.502 \pm 0.397                &
      -28.497 \pm 0.212                &
      -28.405 \pm 0.095                &
      -28.520 \pm 0.175                &
      -28.535 \pm 0.064                &
      -28.529 \pm 0.025                  \\
      (2)                              &
      -28.399 \pm 0.319                &
      -28.454 \pm 0.177                &
      -28.391 \pm 0.080                &
      -28.522 \pm 0.143                &
      -28.537 \pm 0.052                &
      -28.530 \pm 0.021                  \\
      (3)                              &
      -28.377 \pm 0.294                &
      -28.408 \pm 0.139                &
      -28.353 \pm 0.054                &
      -28.507 \pm 0.065                &
      -28.521 \pm 0.026                &
      -28.527 \pm 0.012                  \\
      \bottomrule
    \end{tabular}
  \end{subtable}
\end{table}

a
\begin{figure}[H]
  \includegraphics[width=\textwidth]{media/diff_evaluation.pdf}
  \caption{a}
  \label{fig:eval_diff}
\end{figure}


\section{conclusion}
